\documentclass{amse}

\numberwithin{equation}{section}

\begin{document}

 \PageNum{1}

 \Volume{2009}{Jun.}{25}{1}
 \OnlineTime{July 1, 2009}
 \DOI{00000000000000}

 \EditorNote{Received December 28, 2006, Accepted
May 23, 2008}

 \AuthorMark{Weiping Li}
 \TitleMark{Granny knots, Square knot}

\title{Commutators with Finite Spectrum II\footnote{}}

\author{Nadia BOUDI}
    {D\'epartement de Math\'ematiques, Universit\'e  Moulay
Ismail,\\ Facult\'e des Sciences B. P. $4010,$ Beni M'hamed,
Meknes, Morocco\\
    E-mail\,$:$ yzlee@bjut.edu.cn}


\maketitle


 \Abstract{The purpose of this
paper is to study derivations $d,d'$ defined on a Banach algebra
$A$ such that the spectrum $ \sigma([dx,d'x])$ is finite for all
$x \in A$. In particular we show that if the algebra is
semisimple, then there exists an element $a$ in the socle of $A$
such that $ [d,d']$ is the inner derivation implemented by $a$.}%
 \Keywords{47B47,
47B48, 47A10}%
 \MRSubClass{47B47,
47B48, 47A10}


\newcommand{\socle}{{\rm Soc}\,}
\newcommand{\radic}{{\rm rad}\,}
\newcommand{\densA}{{\mathscr A}}
\newcommand{\Primi}{{\rm Prim}}
\newcommand{\rg}{{\rm rank}\,}
\newcommand{\dime}{{\rm dim}\,}
\newcommand{\Kern}{{\rm Ker}\,}

\section{Introduction}

%\subsection{References and Citation}

 Let $A$ be an algebra. By $[x,y]$ we denote the commutator
$xy-yx$ of $x,y \in A$.  For each $a \in A$ let $\delta_a$ denote
the inner derivation $\delta_a(x)= [a,x]$. A map $f: A\rightarrow
A$ is called commuting  if $ [f(x),x]=0$ for all $x \in A$. A
well-known theorem of Posner [1] states that if $d$ is a nonzero
commuting derivation on a noncommutative prime ring $A$, then $d =
0$. There are many ways how to extend this result either in the
ring theory or in the Banach algebra theory (see [2] for a full
account). The  classical  Kleinecke--Shirokov theorem
 reminiscents Posner's result; one of its versions can be stated as follows:
 If $d$ is a bounded derivation of a Banach algebra
$A$ and $a\in A$ is such that $[da,a] =0$, then $da$ is
quasi-nilpotent (see for instance [3]).  In [4] Bre\v sar
 showed that if $d$ is a bounded
derivation on a Banach algebra $A$ such that $[dx,x]$ is
quasi-nilpotent for all $x \in A$, then $dA$ is contained in the
Jacobson radical  $\radic A $ of $A$. Furthermore, in [5] it was
shown that the following  conditions are equivalent for a bounded
derivation on a Banach algebra $A$:

 (i)\; $[dx,x]$ has finite spectrum for every $x \in A$,

 (ii)\; $dx$ has finite spectrum for every $x \in A$,

 (iii)\; There exists $a \in A$ such that $ a+  A \in \socle(A/ A) $ and $dx- \delta_a(x) \in  A$ for all $x \in
 A$.

 \noindent It should be pointed out that the second condition has
 been investigated earlier in [6] and~[7].  The proofs use extensively
 the Jacobson density theorem and its generalizations (see [8] and
  [9, Extended Jacobson Density Theorem]).

 In this paper, we continue this line of investigation by
 considering derivations $d,d'$ on a complex Banach algebra $A$
 such that the spectrum $ \sigma([dx,d'x])$ is finite for every $x
 \in A$. We shall give a complete characterization of
such $d$ and $d'$ in the case of primitive algebras. We also
consider the special case where $[dx,d'x]$ is quasi-nilpotent for
every $x \in A$.  It turns out that the recent generalization of
the Jacobson density theorem
 given by Beidar and Bre\v{s}ar~[8, Theorem~5.2] allows us to
 treat the general case. Our main result establishes that for all
 but finitely many primitive ideals $P$ of $A$, $P$ is invariant
 under $d$ and $d'$ and $[d_Px, d_P'x]$ is quasi-nilpotent for all
 $x \in A/P$, where $d_P, d_P'$ denote the induced derivations on
 $A/P$. We also  consider the naturally
arising question  whether our condition implies that $ [dx,d'x] +
 A \in  A/  A$ for every $x \in A$.

 This idea of studying derivations $d,d'$ of an algebra with the
 property that  commutators
  $[dx,d'x]$ satisfy some special relation is by no means a novelty.
  In [10], Lanski gave a characterization of derivations $d,d'$ of a
  prime ring $R$ such that $[dx,d'x]=0$ for every $ x \in R$. A
  considerably more general result was obtained later in [11]
  and [12], where the authors considered additive maps
  $f$ and derivations $d$ of a prime ring $R$ such that $
  [f(x),dx]=0$ for every $x \in R$. \par This paper is organized as
  follows. In the first section, which is purely algebraic, we
  proceed with the study of dense algebras.  The second
  section is devoted to our main results in  general Banach
  algebras.

 Throughout this section, $X$ will be a vector space over
${\mathbb C}$ and $\densA $ will be an algebra of linear operators
on $X$ acting densely on~$X$. Let $d$ be a derivation on $\densA
$. Following the terminology of~[6], we shall say that $d$ is {\it
inner\/} if there is a linear operator $T$ on $X$ such that $dS=
\delta_T (S)$ for every $S\in \densA $. Otherwise, $d$ will be
called {\it outer}. As usual, $I$ denotes the identity mapping
on~$X$ and $ \sharp F$ denotes the cardinality of a set $F$. The
dual of $X$ will be denoted by $X^*$ and we will denote by $u
\otimes f$ the linear operator on $X$ defined for any $u \in X$
and $f \in X^*$ by $(u \otimes f)(x)=f(x)u$ for arbitrary $x \in
X$.  If $ J \subset X$ and $x_1,\ldots ,x_n$ are vectors of $X$, $
\langle J \rangle$ and  $\langle x_1,\ldots ,x_n \rangle$ will
denote respectively their linear spans. Let $T$ be an operator on
$X$. The \textit{ spectrum} of $T$ is $\sigma (T)= \{ \lambda \in
{\mathbb C}: \lambda I- T \mbox { is not invertible } \}.$ The
\textit{ point spectrum} of $T$ is $\sigma_p (T)=\{ \lambda \in
{\mathbb C}: \lambda I - T \mbox { is not injective } \}$.

Blablabla (cf. \cite{ref1}) bla bla bla bla bla bla bla bla bla bla bla bla bla bla bla bla bla bla bla bla bla bla
bla bla bla bla bla bla bla bla bla bla bla bla bla bla bla bla bla bla bla bla bla bla bla bla bla bla bla bla bla
bla bla bla bla bla bla bla bla bla bla bla bla bla bla bla bla bla bla bla bla bla bla bla bla bla bla bla bla bla
bla bla bla bla bla bla bla bla bla bla bla bla bla bla bla bla bla bla bla bla bla bla bla bla bla bla bla bla bla ...

Blablabla (cf. \cite{ref1,ref2,ref3,ref4,ref5}) bla bla bla bla
bla bla bla bla bla bla bla bla bla bla bla bla bla bla bla bla
bla bla bla bla bla bla bla bla bla bla bla bla bla bla bla bla
bla bla bla bla bla bla bla bla bla bla bla bla bla bla bla bla
bla bla bla bla bla bla bla bla bla bla bla bla bla bla bla bla
bla bla bla bla bla bla b   la bla bla bla bla bla bla bla bla bla
bla bla bla bla bla bla bla bla bla bla bla bla bla bla bla bla
bla bla bla bla ...

Blablabla (cf. \cite{ref1,ref5}) bla bla bla bla bla bla bla bla bla bla bla bla bla bla bla bla bla bla bla bla bla
bla bla bla bla bla bla bla bla bla bla bla bla bla bla bla bla bla bla bla bla bla bla bla bla bla bla bla bla bla
bla bla bla bla bla bla bla bla bla bla bla bla bla bla bla bla bla bla bla bla bla bla bla bla bla bla bla bla bla
bla bla bla bla bla bla bla bla bla bla bla bla bla bla bla bla bla bla bla bla bla bla bla bla bla bla bla bla bla ...



\subsection{Table}

\begin{table}
\begin{tabular}{|c|c|c|l|c|}
\hline $P(x)$ & $i$& $(e(1),e(2),e(4))$ & $(e(3),e(6),e(12),e(24))$ & $T(E)$ \\
\hline $P_1$  &    & & &$\varnothing$ \\
\hline $P_2$  & 4  & & $(1,1,1,0)\rightarrow(0,0,0,1)$ &2\\
\hline $P_3$  & 2  & &$(1,1,1,0)\rightarrow(0,0,2,0)$ &1\\
\hline $P_4$  & 2  & $(0,1,1)\rightarrow(1,2,0)$ & &1\\
\hline $P_5$  & 2  & $(0,1,1)\rightarrow(1,2,0)$ &$(1,1,1,0)\rightarrow(0,0,0,1)$ &1,2\\
\hline $P_6$  & 6  & $(0,1,1)\rightarrow(1,2,0)$ &$(1,1,1,0)\rightarrow(2,2,0,0)$ &1\\
\hline $P_7$  & 3  & $(0,1,1)\rightarrow(1,0,1)$ &$(1,1,1,0)\rightarrow(2,0,1,0)$ &0\\
\hline $P_8$  & 3  & $(0,1,1)\rightarrow(2,1,0)$ &$(1,1,1,0)\rightarrow(2,0,1,0) \rightarrow(3,1,0,0)$ &0,1\\
\hline
\end{tabular}
\caption{distribution under the null hypothesis with}
\end{table}


\subsection{Figure}

\begin{figure}
\includegraphics[scale=1.2]{actmark.eps}
\caption{Q-Q plots of $2\xi_n$ and extreme distribution under the
null hypothesis with sample sizes 200 and 400 when error
distributions are standard normal distribution $N(0,1)$ and standard
$t_2$ distribution}
\end{figure}


\section{Theorems}

\begin{thm}
Let $f\in L^{2}(\mathbb R^{n})$ and $d \geq
0$. Then
\begin{equation}
\int_{\mathbb R^{n}} \int_{\mathbb R^{n}} \frac{|f(x)|\,
|\mathscr{F}{f}(y)|}{(1+|x|+|y|)^{d}}\, {\rm e}^{|x||y|}\,dxdy <
\infty
\end{equation}
implies that
$$
f(x)=P(x){\rm e}^{-\langle Ax,x\rangle},
$$
where $A$ is a real positive definite symmetric matrix and $P$ is
a polynomial of degree $<\frac{d-n}{2}$. In particular, for $d\leq
n$, the function $f$ is identically $0.$
\end{thm}

\begin{proof}
As shown in ...

Bla bla bla bla bla bla bla bla bla bla bla bla bla bla bla bla bla bla bla bla bla bla
bla bla bla bla bla bla bla bla bla bla bla bla bla bla bla bla bla bla bla bla bla bla bla bla bla bla bla bla bla
bla bla bla bla bla bla bla bla bla bla bla bla bla bla bla bla bla bla bla bla bla bla bla bla bla bla bla bla bla
bla bla bla bla bla bla bla bla bla bla bla bla bla bla bla bla bla bla bla bla bla bla bla bla bla bla bla bla bla ...
\end{proof}



\begin{lem}[Someone's Theorem \cite{ref2}]
Let $f\in L^{2}(\mathbb R^{n})$ and $d \geq
0$. Then
\begin{equation}
\int_{\mathbb R^{n}} \int_{\mathbb R^{n}} \frac{|f(x)|\,
|\mathscr{F}{f}(y)|}{(1+|x|+|y|)^{d}}\, {\rm e}^{|x||y|}\,dxdy <
\infty
\end{equation}
implies that
$$
f(x)=P(x){\rm e}^{-\langle Ax,x\rangle},
$$
where $A$ is a real positive definite symmetric matrix and $P$ is
a polynomial of degree $<\frac{d-n}{2}$. In particular, for $d\leq
n$, the function $f$ is identically $0.$
\end{lem}


\begin{prop}
Let $f\in L^{2}(\mathbb R^{n})$ and $d \geq
0$. Then
\begin{equation}
\int_{\mathbb R^{n}} \int_{\mathbb R^{n}} \frac{|f(x)|\,
|\mathscr{F}{f}(y)|}{(1+|x|+|y|)^{d}}\, {\rm e}^{|x||y|}\,dxdy <
\infty
\end{equation}
implies that
$$
f(x)=P(x){\rm e}^{-\langle Ax,x\rangle},
$$
where $A$ is a real positive definite symmetric matrix and $P$ is
a polynomial of degree $<\frac{d-n}{2}$. In particular, for $d\leq
n$, the function $f$ is identically $0.$
\end{prop}


\begin{cor}
Let $f\in L^{2}(\mathbb R^{n})$ and $d \geq
0$. Then
\begin{equation}
\int_{\mathbb R^{n}} \int_{\mathbb R^{n}} \frac{|f(x)|\,
|\mathscr{F}{f}(y)|}{(1+|x|+|y|)^{d}}\, {\rm e}^{|x||y|}\,dxdy <
\infty
\end{equation}
implies that
$$
f(x)=P(x){\rm e}^{-\langle Ax,x\rangle},
$$
where $A$ is a real positive definite symmetric matrix and $P$ is
a polynomial of degree $<\frac{d-n}{2}$. In particular, for $d\leq
n$, the function $f$ is identically $0.$
\end{cor}


\section{Algorithm}

\begin{algo}
Let $f\in L^{2}(\mathbb R^{n})$ and $d \geq
0$. Then
\begin{equation}
\int_{\mathbb R^{n}} \int_{\mathbb R^{n}} \frac{|f(x)|\,
|\mathscr{F}{f}(y)|}{(1+|x|+|y|)^{d}}\, {\rm e}^{|x||y|}\,dxdy <
\infty
\end{equation}
implies that
$$
f(x)=P(x){\rm e}^{-\langle Ax,x\rangle},
$$
where $A$ is a real positive definite symmetric matrix and $P$ is
a polynomial of degree $<\frac{d-n}{2}$. In particular, for $d\leq
n$, the function $f$ is identically $0.$
\end{algo}


\section{Definitions}

\begin{defn}
Let $f\in L^{2}(\mathbb R^{n})$ and $d \geq
0$. Then
\begin{equation}
\int_{\mathbb R^{n}} \int_{\mathbb R^{n}} \frac{|f(x)|\,
|\mathscr{F}{f}(y)|}{(1+|x|+|y|)^{d}}\, {\rm e}^{|x||y|}\,dxdy <
\infty
\end{equation}
implies that
$$
f(x)=P(x){\rm e}^{-\langle Ax,x\rangle},
$$
where $A$ is a real positive definite symmetric matrix and $P$ is
a polynomial of degree $<\frac{d-n}{2}$. In particular, for $d\leq
n$, the function $f$ is identically $0.$
\end{defn}


\begin{conj}
Let $f\in L^{2}(\mathbb R^{n})$ and $d \geq
0$. Then
\begin{equation}
\int_{\mathbb R^{n}} \int_{\mathbb R^{n}} \frac{|f(x)|\,
|\mathscr{F}{f}(y)|}{(1+|x|+|y|)^{d}}\, {\rm e}^{|x||y|}\,dxdy <
\infty
\end{equation}
implies that
$$
f(x)=P(x){\rm e}^{-\langle Ax,x\rangle},
$$
where $A$ is a real positive definite symmetric matrix and $P$ is
a polynomial of degree $<\frac{d-n}{2}$. In particular, for $d\leq
n$, the function $f$ is identically $0.$
\end{conj}


\begin{exmp}
Let $f\in L^{2}(\mathbb R^{n})$ and $d \geq
0$. Then
\begin{equation}
\int_{\mathbb R^{n}} \int_{\mathbb R^{n}} \frac{|f(x)|\,
|\mathscr{F}{f}(y)|}{(1+|x|+|y|)^{d}}\, {\rm e}^{|x||y|}\,dxdy <
\infty
\end{equation}
implies that
$$
f(x)=P(x){\rm e}^{-\langle Ax,x\rangle},
$$
where $A$ is a real positive definite symmetric matrix and $P$ is
a polynomial of degree $<\frac{d-n}{2}$. In particular, for $d\leq
n$, the function $f$ is identically $0.$
\end{exmp}


\begin{rem}
Let $f\in L^{2}(\mathbb R^{n})$ and $d \geq
0$. Then
\begin{equation}
\int_{\mathbb R^{n}} \int_{\mathbb R^{n}} \frac{|f(x)|\,
|\mathscr{F}{f}(y)|}{(1+|x|+|y|)^{d}}\, {\rm e}^{|x||y|}\,dxdy <
\infty
\end{equation}
implies that
$$
f(x)=P(x){\rm e}^{-\langle Ax,x\rangle},
$$
where $A$ is a real positive definite symmetric matrix and $P$ is
a polynomial of degree $<\frac{d-n}{2}$. In particular, for $d\leq
n$, the function $f$ is identically $0.$
\end{rem}


\begin{case}
Let $f\in L^{2}(\mathbb R^{n})$ and $d \geq
0$. Then
\begin{equation}
\int_{\mathbb R^{n}} \int_{\mathbb R^{n}} \frac{|f(x)|\,
|\mathscr{F}{f}(y)|}{(1+|x|+|y|)^{d}}\, {\rm e}^{|x||y|}\,dxdy <
\infty
\end{equation}
implies that
$$
f(x)=P(x){\rm e}^{-\langle Ax,x\rangle},
$$
where $A$ is a real positive definite symmetric matrix and $P$ is
a polynomial of degree $<\frac{d-n}{2}$. In particular, for $d\leq
n$, the function $f$ is identically $0.$
\end{case}




\acknowledgement{Thanks ....}



\BeginRef

\REF{ref1} Test.

\REF{ref2} Test.

\REF{ref3} Test.

\REF{ref4} Test.

\REF{ref5} Test.

\REF{ref6} Test.

\REF{ref7} Test.

\REF{ref8} Test.

\REF{ref9} Test.

\REF{ref10} Test.

\REF{another-ref1} Test.

\REF{another-ref2} Test.

\REF{another-ref3} Test.

\REF{another-ref4} Test.

\REF{another-ref5} Test.

\EndRef


\end{document}
